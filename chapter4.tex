\documentclass[]{article}
\usepackage{lmodern}
\usepackage{amssymb,amsmath}
\usepackage{ifxetex,ifluatex}
\usepackage{fixltx2e} % provides \textsubscript
\ifnum 0\ifxetex 1\fi\ifluatex 1\fi=0 % if pdftex
  \usepackage[T1]{fontenc}
  \usepackage[utf8]{inputenc}
\else % if luatex or xelatex
  \ifxetex
    \usepackage{mathspec}
  \else
    \usepackage{fontspec}
  \fi
  \defaultfontfeatures{Ligatures=TeX,Scale=MatchLowercase}
\fi
% use upquote if available, for straight quotes in verbatim environments
\IfFileExists{upquote.sty}{\usepackage{upquote}}{}
% use microtype if available
\IfFileExists{microtype.sty}{%
\usepackage{microtype}
\UseMicrotypeSet[protrusion]{basicmath} % disable protrusion for tt fonts
}{}
\usepackage[margin=1in]{geometry}
\usepackage{hyperref}
\hypersetup{unicode=true,
            pdftitle={chapter4},
            pdfauthor={sehrish Naveed},
            pdfborder={0 0 0},
            breaklinks=true}
\urlstyle{same}  % don't use monospace font for urls
\usepackage{color}
\usepackage{fancyvrb}
\newcommand{\VerbBar}{|}
\newcommand{\VERB}{\Verb[commandchars=\\\{\}]}
\DefineVerbatimEnvironment{Highlighting}{Verbatim}{commandchars=\\\{\}}
% Add ',fontsize=\small' for more characters per line
\usepackage{framed}
\definecolor{shadecolor}{RGB}{248,248,248}
\newenvironment{Shaded}{\begin{snugshade}}{\end{snugshade}}
\newcommand{\KeywordTok}[1]{\textcolor[rgb]{0.13,0.29,0.53}{\textbf{#1}}}
\newcommand{\DataTypeTok}[1]{\textcolor[rgb]{0.13,0.29,0.53}{#1}}
\newcommand{\DecValTok}[1]{\textcolor[rgb]{0.00,0.00,0.81}{#1}}
\newcommand{\BaseNTok}[1]{\textcolor[rgb]{0.00,0.00,0.81}{#1}}
\newcommand{\FloatTok}[1]{\textcolor[rgb]{0.00,0.00,0.81}{#1}}
\newcommand{\ConstantTok}[1]{\textcolor[rgb]{0.00,0.00,0.00}{#1}}
\newcommand{\CharTok}[1]{\textcolor[rgb]{0.31,0.60,0.02}{#1}}
\newcommand{\SpecialCharTok}[1]{\textcolor[rgb]{0.00,0.00,0.00}{#1}}
\newcommand{\StringTok}[1]{\textcolor[rgb]{0.31,0.60,0.02}{#1}}
\newcommand{\VerbatimStringTok}[1]{\textcolor[rgb]{0.31,0.60,0.02}{#1}}
\newcommand{\SpecialStringTok}[1]{\textcolor[rgb]{0.31,0.60,0.02}{#1}}
\newcommand{\ImportTok}[1]{#1}
\newcommand{\CommentTok}[1]{\textcolor[rgb]{0.56,0.35,0.01}{\textit{#1}}}
\newcommand{\DocumentationTok}[1]{\textcolor[rgb]{0.56,0.35,0.01}{\textbf{\textit{#1}}}}
\newcommand{\AnnotationTok}[1]{\textcolor[rgb]{0.56,0.35,0.01}{\textbf{\textit{#1}}}}
\newcommand{\CommentVarTok}[1]{\textcolor[rgb]{0.56,0.35,0.01}{\textbf{\textit{#1}}}}
\newcommand{\OtherTok}[1]{\textcolor[rgb]{0.56,0.35,0.01}{#1}}
\newcommand{\FunctionTok}[1]{\textcolor[rgb]{0.00,0.00,0.00}{#1}}
\newcommand{\VariableTok}[1]{\textcolor[rgb]{0.00,0.00,0.00}{#1}}
\newcommand{\ControlFlowTok}[1]{\textcolor[rgb]{0.13,0.29,0.53}{\textbf{#1}}}
\newcommand{\OperatorTok}[1]{\textcolor[rgb]{0.81,0.36,0.00}{\textbf{#1}}}
\newcommand{\BuiltInTok}[1]{#1}
\newcommand{\ExtensionTok}[1]{#1}
\newcommand{\PreprocessorTok}[1]{\textcolor[rgb]{0.56,0.35,0.01}{\textit{#1}}}
\newcommand{\AttributeTok}[1]{\textcolor[rgb]{0.77,0.63,0.00}{#1}}
\newcommand{\RegionMarkerTok}[1]{#1}
\newcommand{\InformationTok}[1]{\textcolor[rgb]{0.56,0.35,0.01}{\textbf{\textit{#1}}}}
\newcommand{\WarningTok}[1]{\textcolor[rgb]{0.56,0.35,0.01}{\textbf{\textit{#1}}}}
\newcommand{\AlertTok}[1]{\textcolor[rgb]{0.94,0.16,0.16}{#1}}
\newcommand{\ErrorTok}[1]{\textcolor[rgb]{0.64,0.00,0.00}{\textbf{#1}}}
\newcommand{\NormalTok}[1]{#1}
\usepackage{graphicx,grffile}
\makeatletter
\def\maxwidth{\ifdim\Gin@nat@width>\linewidth\linewidth\else\Gin@nat@width\fi}
\def\maxheight{\ifdim\Gin@nat@height>\textheight\textheight\else\Gin@nat@height\fi}
\makeatother
% Scale images if necessary, so that they will not overflow the page
% margins by default, and it is still possible to overwrite the defaults
% using explicit options in \includegraphics[width, height, ...]{}
\setkeys{Gin}{width=\maxwidth,height=\maxheight,keepaspectratio}
\IfFileExists{parskip.sty}{%
\usepackage{parskip}
}{% else
\setlength{\parindent}{0pt}
\setlength{\parskip}{6pt plus 2pt minus 1pt}
}
\setlength{\emergencystretch}{3em}  % prevent overfull lines
\providecommand{\tightlist}{%
  \setlength{\itemsep}{0pt}\setlength{\parskip}{0pt}}
\setcounter{secnumdepth}{0}
% Redefines (sub)paragraphs to behave more like sections
\ifx\paragraph\undefined\else
\let\oldparagraph\paragraph
\renewcommand{\paragraph}[1]{\oldparagraph{#1}\mbox{}}
\fi
\ifx\subparagraph\undefined\else
\let\oldsubparagraph\subparagraph
\renewcommand{\subparagraph}[1]{\oldsubparagraph{#1}\mbox{}}
\fi

%%% Use protect on footnotes to avoid problems with footnotes in titles
\let\rmarkdownfootnote\footnote%
\def\footnote{\protect\rmarkdownfootnote}

%%% Change title format to be more compact
\usepackage{titling}

% Create subtitle command for use in maketitle
\providecommand{\subtitle}[1]{
  \posttitle{
    \begin{center}\large#1\end{center}
    }
}

\setlength{\droptitle}{-2em}

  \title{chapter4}
    \pretitle{\vspace{\droptitle}\centering\huge}
  \posttitle{\par}
    \author{sehrish Naveed}
    \preauthor{\centering\large\emph}
  \postauthor{\par}
      \predate{\centering\large\emph}
  \postdate{\par}
    \date{25 marraskuuta 2019}


\begin{document}
\maketitle

\subsection{R Markdown}\label{r-markdown}

This is an R Markdown document. Markdown is a simple formatting syntax
for authoring HTML, PDF, and MS Word documents. For more details on
using R Markdown see \url{http://rmarkdown.rstudio.com}.

When you click the \textbf{Knit} button a document will be generated
that includes both content as well as the output of any embedded R code
chunks within the document. You can embed an R code chunk like this:

\begin{Shaded}
\begin{Highlighting}[]
\KeywordTok{summary}\NormalTok{(cars)}
\end{Highlighting}
\end{Shaded}

\begin{verbatim}
##      speed           dist       
##  Min.   : 4.0   Min.   :  2.00  
##  1st Qu.:12.0   1st Qu.: 26.00  
##  Median :15.0   Median : 36.00  
##  Mean   :15.4   Mean   : 42.98  
##  3rd Qu.:19.0   3rd Qu.: 56.00  
##  Max.   :25.0   Max.   :120.00
\end{verbatim}

\subsection{Including Plots}\label{including-plots}

You can also embed plots, for example:

\includegraphics{chapter4_files/figure-latex/pressure-1.pdf}

Note that the \texttt{echo\ =\ FALSE} parameter was added to the code
chunk to prevent printing of the R code that generated the plot.

\section{access the MASS package}\label{access-the-mass-package}

library(MASS)

\section{load the data}\label{load-the-data}

data(``Boston'')

\section{explore the dataset}\label{explore-the-dataset}

str(Boston)

summary(Boston)

\subsection{\texorpdfstring{`data.frame': 506 obs. of 14 variables. The
variables are as
follows:}{data.frame: 506 obs. of 14 variables. The variables are as follows:}}\label{data.frame-506-obs.-of-14-variables.-the-variables-are-as-follows}

\$ crim : num 0.00632 0.02731 0.02729 0.03237 0.06905 \ldots{}

\$ zn : num 18 0 0 0 0 0 12.5 12.5 12.5 12.5 \ldots{}

\$ indus : num 2.31 7.07 7.07 2.18 2.18 2.18 7.87 7.87 7.87 7.87
\ldots{}

\$ chas : int 0 0 0 0 0 0 0 0 0 0 \ldots{}

\$ nox : num 0.538 0.469 0.469 0.458 0.458 0.458 0.524 0.524 0.524 0.524
\ldots{}

\$ rm : num 6.58 6.42 7.18 7 7.15 \ldots{}

\$ age : num 65.2 78.9 61.1 45.8 54.2 58.7 66.6 96.1 100 85.9 \ldots{}

\$ dis : num 4.09 4.97 4.97 6.06 6.06 \ldots{}

\$ rad : int 1 2 2 3 3 3 5 5 5 5 \ldots{}

\$ tax : num 296 242 242 222 222 222 311 311 311 311 \ldots{}

\$ ptratio: num 15.3 17.8 17.8 18.7 18.7 18.7 15.2 15.2 15.2 15.2
\ldots{}

\$ black : num 397 397 393 395 397 \ldots{}

\$ lstat : num 4.98 9.14 4.03 2.94 5.33 \ldots{}

\$ medv : num 24 21.6 34.7 33.4 36.2 28.7 22.9 27.1 16.5 18.9 \ldots{}

\subsection{Below is the summary of the
data}\label{below-is-the-summary-of-the-data}

\begin{quote}
summary(Boston)
\end{quote}

\begin{verbatim}
  crim                zn             indus            chas         
\end{verbatim}

Min. : 0.00632 Min. : 0.00 Min. : 0.46 Min. :0.00000

1st Qu.: 0.08204 1st Qu.: 0.00 1st Qu.: 5.19 1st Qu.:0.00000

Median : 0.25651 Median : 0.00 Median : 9.69 Median :0.00000

Mean : 3.61352 Mean : 11.36 Mean :11.14 Mean :0.06917

3rd Qu.: 3.67708 3rd Qu.: 12.50 3rd Qu.:18.10 3rd Qu.:0.00000

Max. :88.97620 Max. :100.00 Max. :27.74 Max. :1.00000

\begin{verbatim}
  nox               rm             age              dis         
\end{verbatim}

Min. :0.3850 Min. :3.561 Min. : 2.90 Min. : 1.130

1st Qu.:0.4490 1st Qu.:5.886 1st Qu.: 45.02 1st Qu.: 2.100

Median :0.5380 Median :6.208 Median : 77.50 Median : 3.207

Mean :0.5547 Mean :6.285 Mean : 68.57 Mean : 3.795

3rd Qu.:0.6240 3rd Qu.:6.623 3rd Qu.: 94.08 3rd Qu.: 5.188

Max. :0.8710 Max. :8.780 Max. :100.00 Max. :12.127

\begin{verbatim}
  rad              tax           ptratio          black        
\end{verbatim}

Min. : 1.000 Min. :187.0 Min. :12.60 Min. : 0.32

1st Qu.: 4.000 1st Qu.:279.0 1st Qu.:17.40 1st Qu.:375.38

Median : 5.000 Median :330.0 Median :19.05 Median :391.44

Mean : 9.549 Mean :408.2 Mean :18.46 Mean :356.67

3rd Qu.:24.000 3rd Qu.:666.0 3rd Qu.:20.20 3rd Qu.:396.23

Max. :24.000 Max. :711.0 Max. :22.00 Max. :396.90

\begin{verbatim}
 lstat            medv       
\end{verbatim}

Min. : 1.73 Min. : 5.00

1st Qu.: 6.95 1st Qu.:17.02

Median :11.36 Median :21.20

Mean :12.65 Mean :22.53

3rd Qu.:16.95 3rd Qu.:25.00

Max. :37.97 Max. :50.00

\begin{quote}
\end{quote}

\begin{quote}
\section{plot matrix of the
variables}\label{plot-matrix-of-the-variables}
\end{quote}

\begin{quote}
pairs(Boston) \# MASS, corrplot, tidyr and Boston dataset are available
\end{quote}

\section{calculate the correlation matrix and round
it}\label{calculate-the-correlation-matrix-and-round-it}

cor\_matrix\textless{}-cor(Boston) \%\textgreater{}\% round(digits = 2)

\section{print the correlation
matrix}\label{print-the-correlation-matrix}

cor\_matrix

\section{visualize the correlation
matrix}\label{visualize-the-correlation-matrix}

corrplot(cor\_matrix, method=``circle'', type=``upper'', cl.pos=``b'',
tl.pos=``d'', tl.cex = 0.6)

\section{MASS, corrplot, tidyr and Boston dataset are
available}\label{mass-corrplot-tidyr-and-boston-dataset-are-available}

\begin{quote}
\end{quote}

\section{calculate the correlation matrix and round
it}\label{calculate-the-correlation-matrix-and-round-it-1}

\begin{quote}
cor\_matrix\textless{}-cor(Boston) \%\textgreater{}\% round(digits = 2)
\end{quote}

\begin{quote}
\end{quote}

\section{print the correlation
matrix}\label{print-the-correlation-matrix-1}

\begin{quote}
cor\_matrix
\end{quote}

\begin{verbatim}
     crim    zn indus  chas   nox    rm   age   dis   rad   tax ptratio 
\end{verbatim}

crim 1.00 -0.20 0.41 -0.06 0.42 -0.22 0.35 -0.38 0.63 0.58 0.29

zn -0.20 1.00 -0.53 -0.04 -0.52 0.31 -0.57 0.66 -0.31 -0.31 -0.39

indus 0.41 -0.53 1.00 0.06 0.76 -0.39 0.64 -0.71 0.60 0.72 0.38

chas -0.06 -0.04 0.06 1.00 0.09 0.09 0.09 -0.10 -0.01 -0.04 -0.12

nox 0.42 -0.52 0.76 0.09 1.00 -0.30 0.73 -0.77 0.61 0.67 0.19

rm -0.22 0.31 -0.39 0.09 -0.30 1.00 -0.24 0.21 -0.21 -0.29 -0.36

age 0.35 -0.57 0.64 0.09 0.73 -0.24 1.00 -0.75 0.46 0.51 0.26

dis -0.38 0.66 -0.71 -0.10 -0.77 0.21 -0.75 1.00 -0.49 -0.53 -0.23

rad 0.63 -0.31 0.60 -0.01 0.61 -0.21 0.46 -0.49 1.00 0.91 0.46

tax 0.58 -0.31 0.72 -0.04 0.67 -0.29 0.51 -0.53 0.91 1.00 0.46

ptratio 0.29 -0.39 0.38 -0.12 0.19 -0.36 0.26 -0.23 0.46 0.46 1.00

black -0.39 0.18 -0.36 0.05 -0.38 0.13 -0.27 0.29 -0.44 -0.44 -0.18

lstat 0.46 -0.41 0.60 -0.05 0.59 -0.61 0.60 -0.50 0.49 0.54 0.37

medv -0.39 0.36 -0.48 0.18 -0.43 0.70 -0.38 0.25 -0.38 -0.47 -0.51

\begin{verbatim}
    black lstat  medv 
\end{verbatim}

crim -0.39 0.46 -0.39

zn 0.18 -0.41 0.36

indus -0.36 0.60 -0.48

chas 0.05 -0.05 0.18

nox -0.38 0.59 -0.43

rm 0.13 -0.61 0.70

age -0.27 0.60 -0.38

dis 0.29 -0.50 0.25

rad -0.44 0.49 -0.38

tax -0.44 0.54 -0.47

ptratio -0.18 0.37 -0.51

black 1.00 -0.37 0.33

lstat -0.37 1.00 -0.74

medv 0.33 -0.74 1.00

\begin{quote}
\end{quote}

\section{visualize the correlation
matrix}\label{visualize-the-correlation-matrix-1}

\begin{quote}
corrplot(cor\_matrix, method=``circle'', type=``upper'', cl.pos=``b'',
tl.pos=``d'', tl.cex = 0.6)
\end{quote}

\section{MASS and Boston dataset are
available}\label{mass-and-boston-dataset-are-available}

\section{center and standardize
variables}\label{center-and-standardize-variables}

boston\_scaled \textless{}- scale(Boston)

\section{summaries of the scaled
variables}\label{summaries-of-the-scaled-variables}

summary(boston\_scaled)

\section{class of the boston\_scaled
object}\label{class-of-the-boston_scaled-object}

class(boston\_scaled)

\section{change the object to data
frame}\label{change-the-object-to-data-frame}

boston\_scaled \textless{}- as.data.frame(boston\_scaled)

\section{MASS and Boston dataset are
available}\label{mass-and-boston-dataset-are-available-1}

\begin{quote}
\end{quote}

\section{center and standardize
variables}\label{center-and-standardize-variables-1}

\begin{quote}
boston\_scaled \textless{}- scale(Boston)
\end{quote}

\begin{quote}
\end{quote}

\section{summaries of the scaled
variables}\label{summaries-of-the-scaled-variables-1}

\begin{quote}
summary(boston\_scaled)
\end{quote}

\begin{verbatim}
  crim                 zn               indus              chas         
\end{verbatim}

Min. :-0.419367 Min. :-0.48724 Min. :-1.5563 Min. :-0.2723

1st Qu.:-0.410563 1st Qu.:-0.48724 1st Qu.:-0.8668 1st Qu.:-0.2723

Median :-0.390280 Median :-0.48724 Median :-0.2109 Median :-0.2723

Mean : 0.000000 Mean : 0.00000 Mean : 0.0000 Mean : 0.0000

3rd Qu.: 0.007389 3rd Qu.: 0.04872 3rd Qu.: 1.0150 3rd Qu.:-0.2723

Max. : 9.924110 Max. : 3.80047 Max. : 2.4202 Max. : 3.6648

\begin{verbatim}
  nox                rm               age               dis          
\end{verbatim}

Min. :-1.4644 Min. :-3.8764 Min. :-2.3331 Min. :-1.2658

1st Qu.:-0.9121 1st Qu.:-0.5681 1st Qu.:-0.8366 1st Qu.:-0.8049

Median :-0.1441 Median :-0.1084 Median : 0.3171 Median :-0.2790

Mean : 0.0000 Mean : 0.0000 Mean : 0.0000 Mean : 0.0000

3rd Qu.: 0.5981 3rd Qu.: 0.4823 3rd Qu.: 0.9059 3rd Qu.: 0.6617

Max. : 2.7296 Max. : 3.5515 Max. : 1.1164 Max. : 3.9566

\begin{verbatim}
  rad               tax             ptratio            black         
\end{verbatim}

Min. :-0.9819 Min. :-1.3127 Min. :-2.7047 Min. :-3.9033

1st Qu.:-0.6373 1st Qu.:-0.7668 1st Qu.:-0.4876 1st Qu.: 0.2049

Median :-0.5225 Median :-0.4642 Median : 0.2746 Median : 0.3808

Mean : 0.0000 Mean : 0.0000 Mean : 0.0000 Mean : 0.0000

3rd Qu.: 1.6596 3rd Qu.: 1.5294 3rd Qu.: 0.8058 3rd Qu.: 0.4332

Max. : 1.6596 Max. : 1.7964 Max. : 1.6372 Max. : 0.4406

\begin{verbatim}
 lstat              medv         
\end{verbatim}

Min. :-1.5296 Min. :-1.9063

1st Qu.:-0.7986 1st Qu.:-0.5989

Median :-0.1811 Median :-0.1449

Mean : 0.0000 Mean : 0.0000

3rd Qu.: 0.6024 3rd Qu.: 0.2683

Max. : 3.5453 Max. : 2.9865

\begin{quote}
\end{quote}

\section{class of the boston\_scaled
object}\label{class-of-the-boston_scaled-object-1}

\begin{quote}
class(boston\_scaled)
\end{quote}

{[}1{]} ``matrix''

\begin{quote}
\end{quote}

\section{change the object to data
frame}\label{change-the-object-to-data-frame-1}

\begin{quote}
boston\_scaled \textless{}- as.data.frame(boston\_scaled)
\end{quote}

\begin{quote}
\end{quote}

\subsection{creating factor variables}\label{creating-factor-variables}

\section{MASS, Boston and boston\_scaled are
available}\label{mass-boston-and-boston_scaled-are-available}

\section{summary of the scaled crime
rate}\label{summary-of-the-scaled-crime-rate}

summary(boston\_scaled\$crim)

\section{create a quantile vector of crim and print
it}\label{create-a-quantile-vector-of-crim-and-print-it}

bins \textless{}- quantile(boston\_scaled\$crim)

bins

\section{\texorpdfstring{create a categorical variable
`crime'}{create a categorical variable crime}}\label{create-a-categorical-variable-crime}

crime \textless{}- cut(boston\_scaled\$crim, breaks = bins,
include.lowest = TRUE, labels = c(``low'', ``med\_low'', ``med\_high'',
``high''))

\section{look at the table of the new factor
crime}\label{look-at-the-table-of-the-new-factor-crime}

table(crime)

\section{remove original crim from the
dataset}\label{remove-original-crim-from-the-dataset}

boston\_scaled \textless{}- dplyr::select(boston\_scaled, -crim)

\section{add the new categorical value to scaled
data}\label{add-the-new-categorical-value-to-scaled-data}

boston\_scaled \textless{}- data.frame(boston\_scaled, crime)

\section{MASS, Boston and boston\_scaled are
available}\label{mass-boston-and-boston_scaled-are-available-1}

\begin{quote}
\end{quote}

\section{summary of the scaled crime
rate}\label{summary-of-the-scaled-crime-rate-1}

\begin{quote}
summary(boston\_scaled\$crim)
\end{quote}

\begin{verbatim}
 Min.   1st Qu.    Median      Mean   3rd Qu.      Max.  
\end{verbatim}

-0.419367 -0.410563 -0.390280 0.000000 0.007389 9.924110

\begin{quote}
\end{quote}

\section{create a quantile vector of crim and print
it}\label{create-a-quantile-vector-of-crim-and-print-it-1}

\begin{quote}
bins \textless{}- quantile(boston\_scaled\$crim)
\end{quote}

\begin{quote}
bins
\end{quote}

\begin{verbatim}
      0%          25%          50%          75%         100%  
\end{verbatim}

-0.419366929 -0.410563278 -0.390280295 0.007389247 9.924109610

\begin{quote}
\end{quote}

\section{\texorpdfstring{create a categorical variable
`crime'}{create a categorical variable crime}}\label{create-a-categorical-variable-crime-1}

\begin{quote}
crime \textless{}- cut(boston\_scaled\$crim, breaks = bins,
include.lowest = TRUE, labels = c(``low'', ``med\_low'', ``med\_high'',
``high''))
\end{quote}

\begin{quote}
\end{quote}

\section{look at the table of the new factor
crime}\label{look-at-the-table-of-the-new-factor-crime-1}

\begin{quote}
table(crime)
\end{quote}

crime

\begin{verbatim}
 low  med_low med_high     high  

 127      126      126      127 
\end{verbatim}

\begin{quote}
\end{quote}

\section{remove original crim from the
dataset}\label{remove-original-crim-from-the-dataset-1}

\begin{quote}
boston\_scaled \textless{}- dplyr::select(boston\_scaled, -crim)
\end{quote}

\begin{quote}
\end{quote}

\section{add the new categorical value to scaled
data}\label{add-the-new-categorical-value-to-scaled-data-1}

\begin{quote}
boston\_scaled \textless{}- data.frame(boston\_scaled, crime)
\end{quote}

\section{boston\_scaled is available}\label{boston_scaled-is-available}

\begin{quote}
\end{quote}

\section{number of rows in the Boston
dataset}\label{number-of-rows-in-the-boston-dataset}

\begin{quote}
n \textless{}- nrow(boston\_scaled)
\end{quote}

\begin{quote}
\end{quote}

\section{choose randomly 80\% of the
rows}\label{choose-randomly-80-of-the-rows}

\begin{quote}
ind \textless{}- sample(n, size = n * 0.8)
\end{quote}

\begin{quote}
\end{quote}

\section{create train set}\label{create-train-set}

\begin{quote}
train \textless{}- boston\_scaled{[}ind,{]}
\end{quote}

\begin{quote}
\end{quote}

\section{create test set}\label{create-test-set}

\begin{quote}
test \textless{}- boston\_scaled{[}-ind,{]}
\end{quote}

\begin{quote}
\end{quote}

\section{save the correct classes from test
data}\label{save-the-correct-classes-from-test-data}

\begin{quote}
correct\_classes \textless{}- test\$crime
\end{quote}

\begin{quote}
\end{quote}

\section{remove the crime variable from test
data}\label{remove-the-crime-variable-from-test-data}

\begin{quote}
test \textless{}- dplyr::select(test, -crime)
\end{quote}

\section{MASS and train are
available}\label{mass-and-train-are-available}

\begin{quote}
\end{quote}

\section{linear discriminant
analysis}\label{linear-discriminant-analysis}

\begin{quote}
lda.fit \textless{}- lda(crime \textasciitilde{} ., data = train)
\end{quote}

\begin{quote}
\end{quote}

\section{print the lda.fit object}\label{print-the-lda.fit-object}

\begin{quote}
lda.fit
\end{quote}

Call:

lda(crime \textasciitilde{} ., data = train)

Prior probabilities of groups:

\begin{verbatim}
 high       low  med_high   med_low  
\end{verbatim}

0.2648515 0.2500000 0.2475248 0.2376238

Group means:

\begin{verbatim}
              zn      indus        chas        nox         rm        age 
\end{verbatim}

high -0.48724019 1.0170108 -0.05155709 1.0874849 -0.4499583 0.8166814

low 0.95232783 -0.9250089 -0.11640431 -0.8866640 0.4352599 -0.8767268

med\_high -0.39205294 0.2882390 0.08200995 0.4187600 0.0769062 0.3947975

med\_low -0.03859348 -0.3051361 -0.02626030 -0.5735058 -0.0890258
-0.3407444

\begin{verbatim}
            dis        rad        tax     ptratio       black       lstat 
\end{verbatim}

high -0.8667357 1.6392096 1.5148289 0.78203563 -0.83354346 0.89888497

low 0.9055335 -0.6919117 -0.7331553 -0.47063537 0.37879901 -0.78335053

med\_high -0.4094969 -0.3973015 -0.2641979 -0.28801353 0.05271565
0.00069132

med\_low 0.4131974 -0.5308586 -0.4600722 -0.08868211 0.32127617
-0.15667032

\begin{verbatim}
             medv 
\end{verbatim}

high -0.713664582

low 0.513169023

med\_high 0.150503209

med\_low 0.007645741

Coefficients of linear discriminants:

\begin{verbatim}
            LD1          LD2          LD3 
\end{verbatim}

zn -0.12831723 -0.543423723 -0.972710885

indus -0.01175966 0.438866847 0.271907394

chas 0.06902287 -0.017458922 0.201348842

nox -0.41140906 0.737732653 -1.272904819

rm 0.07835835 0.079248455 -0.097322468

age -0.26323203 0.224837026 -0.005725033

dis 0.11862801 0.101478612 0.350995081

rad -3.00587168 -0.900636379 -0.080274703

tax 0.01924470 -0.148583571 0.522719019

ptratio -0.11031478 -0.009431194 -0.233849275

black 0.13776978 -0.030198922 0.152095214

lstat -0.22536208 0.137227494 0.443168317

medv -0.18955773 0.351250475 -0.189255741

Proportion of trace:

LD1 LD2 LD3

0.9488 0.0397 0.0116

\begin{quote}
\end{quote}

\section{the function for lda biplot
arrows}\label{the-function-for-lda-biplot-arrows}

\begin{quote}
lda.arrows \textless{}- function(x, myscale = 1, arrow\_heads = 0.1,
color = ``orange'', tex = 0.75, choices = c(1,2))\{
\end{quote}

\begin{verbatim}
heads <- coef(x) 

arrows(x0 = 0, y0 = 0,  

       x1 = myscale * heads[,choices[1]],  

       y1 = myscale * heads[,choices[2]], col=color, length = arrow_heads) 

text(myscale * heads[,choices], labels = row.names(heads),  

     cex = tex, col=color, pos=3) 
\end{verbatim}

\}

\begin{quote}
\end{quote}

\section{target classes as numeric}\label{target-classes-as-numeric}

\begin{quote}
classes \textless{}- as.numeric(train\$crime)
\end{quote}

\begin{quote}
\end{quote}

\section{plot the lda results}\label{plot-the-lda-results}

\begin{quote}
plot(lda.fit, dimen = 2, col = classes, pch = classes)
\end{quote}

\begin{quote}
lda.arrows(lda.fit, myscale = 1)
\end{quote}

\section{lda.fit, correct\_classes and test are
available}\label{lda.fit-correct_classes-and-test-are-available}

\begin{quote}
\end{quote}

\section{predict classes with test
data}\label{predict-classes-with-test-data}

\begin{quote}
lda.pred \textless{}- predict(lda.fit, newdata = test)
\end{quote}

\begin{quote}
\end{quote}

\section{cross tabulate the results}\label{cross-tabulate-the-results}

\begin{quote}
table(correct = correct\_classes, predicted = lda.pred\$class)
\end{quote}

\begin{verbatim}
      predicted 
\end{verbatim}

correct high low med\_high med\_low

high 25 0 0 0

low 0 14 0 7

med\_high 1 1 15 3

med\_low 0 7 10 19

\section{load MASS and Boston}\label{load-mass-and-boston}

\begin{quote}
library(MASS)
\end{quote}

\begin{quote}
data(`Boston')
\end{quote}

\begin{quote}
\end{quote}

\section{euclidean distance matrix}\label{euclidean-distance-matrix}

\begin{quote}
dist\_eu \textless{}- dist(Boston)
\end{quote}

\begin{quote}
\end{quote}

\section{look at the summary of the
distances}\label{look-at-the-summary-of-the-distances}

\begin{quote}
summary(dist\_eu)
\end{quote}

Min. 1st Qu. Median Mean 3rd Qu. Max.

1.119 85.624 170.539 226.315 371.950 626.047

\section{manhattan distance matrix}\label{manhattan-distance-matrix}

\begin{quote}
dist\_man \textless{}- dist(Boston, method = `manhattan')
\end{quote}

\section{look at the summary of the
distances}\label{look-at-the-summary-of-the-distances-1}

\begin{quote}
summary(dist\_man)
\end{quote}

\begin{verbatim}
Min.  1st Qu.   Median     Mean  3rd Qu.     Max.  
\end{verbatim}

2.016 149.145 279.505 342.899 509.707 1198.265

\section{Boston dataset is available}\label{boston-dataset-is-available}

\section{k-means clustering}\label{k-means-clustering}

\begin{quote}
km \textless{}-kmeans(Boston, centers = 3)
\end{quote}

\section{plot the Boston dataset with
clusters}\label{plot-the-boston-dataset-with-clusters}

\begin{quote}
pairs(Boston, col = km\$cluster)
\end{quote}

\section{Boston dataset is
available}\label{boston-dataset-is-available-1}

\begin{quote}
set.seed(123)
\end{quote}

\section{determine the number of
clusters}\label{determine-the-number-of-clusters}

\begin{quote}
k\_max \textless{}- 10
\end{quote}

\section{calculate the total within sum of
squares}\label{calculate-the-total-within-sum-of-squares}

\begin{quote}
twcss \textless{}- sapply(1:k\_max, function(k)\{kmeans(Boston,
k)\$tot.withinss\})
\end{quote}

\section{visualize the results}\label{visualize-the-results}

\begin{quote}
qplot(x = 1:k\_max, y = twcss, geom = `line')
\end{quote}

\section{k-means clustering}\label{k-means-clustering-1}

\begin{quote}
km \textless{}-kmeans(Boston, centers = 2)
\end{quote}

\section{plot the Boston dataset with
clusters}\label{plot-the-boston-dataset-with-clusters-1}

\begin{quote}
pairs(Boston, col = km\$cluster)
\end{quote}


\end{document}
